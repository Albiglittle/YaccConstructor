\documentclass{article}
\usepackage[x11names, rgb]{xcolor}
\usepackage[utf8]{inputenc}
\usepackage{tikz}
\usepackage[russian]{babel}
%\usepackage{dot2texi}
\usetikzlibrary{snakes,arrows,shapes}
\usepackage{amsmath}

\textwidth=160mm
%\textheight=250mm
%\topmargin=-30mm
\oddsidemargin=-5mm
\evensidemargin=-5mm

\begin{document}

\section{Построение TNFA}  

Для построения TNFA по дереву можно использовать несколько изменённый алгоритм Томпсона. Узлы дерва YARD можно разделить на 4 типа: лист, последовательность, альтернатива, замыкание. Эти типы соответствуют основным конструкциям расширенных регулярных выражений.

Атрибут ребра (дуги, перехода в NFA) - функция. В общем случае ей доступна вся информация о состоянии автомата и она может производить любые действия. В нашем случае она бутет записывать информацию о начале и конце распознования конструкции регулярного выражения в список. 

Каждая метка соответствует началу или концу распознования той или иной  конструкции регулярного выражения и содержит:
  \begin{itemize}
    \item уникальный идентификатор, одинаковый для метки начала и конца одной и той же конструкции
    \item информацию о конструкции, которой она соответствует
    \item позицию маркера в строке, в момент генерации метки
  \end{itemize}

Определим следующий набор меток:
  \begin{verbatim}
    type tag = 
      //Последовательность
      | TSeqStart of int*int //(уникальный идентификатор, позиция маркера)
      | TSeqEnd of int*int
      //Первая ветка альтернативы
      | TAlt1Start of int*int
      | TAlt1End of int*int  
      //Вторая ветка альтернативы
      | TAlt2Start of int*int
      | TAlt2End of int*int
      //Замыкание
      | TClsStart of int*int
      | TClsEnd of int*int  
  \end{verbatim}

Определим функции, которые будем использовать в качестве меток для рёбер. Общим параметром для всех функций будет текущая позиция указателя в строке $pos$. Так же каждой функции доступна некоторая общая структура данных, в которую ведётся запись данных, $globalTagsList$. Набор функций выглядит следующим образом:
  \begin{itemize}
    \item 
      $\omega          $ -- \verb|fun _ -> () |  Будет означать отсутствие метки.
    \item 
      $SeqStart_k(pos) $ -- \verb| fun pos -> TSeqStart(k, pos) :: globalTagsList| Подразумевается, что уникальный идентификатор вычисляется заранее, поэтому для разных конструкций строятся разные функции. Чтобы различать их, присвоим им индекс, соответствующий идентификатору.
    \item 
      $SeqEnd_k(pos)   $ -- \verb| fun pos -> TSeqEnd(k, pos) :: globalTagsList|  
    \item 
      $Alt1Start_k(pos)$ -- \verb|fun pos -> TAlt1Start(k, pos) :: globalTagsList|
    \item 
      $Alt1End_k(pos)  $ -- \verb|fun pos -> TAlt1End(k, pos) :: globalTagsList|
    \item 
      $Alt2Start_k(pos)$ -- \verb|fun pos -> TAlt2Start(k, pos) :: globalTagsList|
    \item 
      $Alt2End_k(pos)$   -- \verb|fun pos -> TAlt2End(k, pos) :: globalTagsList|
    \item 
      $ClsStart_k(pos)$  -- \verb|fun pos -> TClsStart(k, pos) :: globalTagsList|
    \item 
      $ClsEnd_k(pos)$    -- \verb|fun pos -> TClsEnd(k, pos) :: globalTagsList|
  \end{itemize}

Тогда для расстановки меток можно будет дополнить алгоритм Томпсона. Результат будет таким:
  \begin{itemize}
    \item
      Лист : Leaf(a) \
      \begin{flushleft}
        \enlargethispage{100cm}
% Start of code
% \begin{tikzpicture}[anchor=mid,>=latex',join=bevel,]
\begin{tikzpicture}[>=latex',join=bevel,]
\pgfsetlinewidth{1bp}
%%
\pgfsetcolor{black}
  % Edge: i_1 -> j_1
  \draw [->] (54bp,18bp) .. controls (68bp,18bp) and (86bp,18bp)  .. (112bp,18bp);
  \draw (83bp,26bp) node {$a/\omega$};
  % Node: j_1
\begin{scope}
  \pgfsetstrokecolor{black}
  \draw (139bp,18bp) node {$i+1$};
\end{scope}
  % Node: i_1
\begin{scope}
  \pgfsetstrokecolor{black}
  \draw (27bp,18bp) node {$i$};
\end{scope}
%
\end{tikzpicture}
% End of code


      \end{flushleft}
    \item 
      Последовательность : Seq(lst) \
      \begin{flushleft}
        \enlargethispage{100cm}
% Start of code
% \begin{tikzpicture}[anchor=mid,>=latex',join=bevel,]
\begin{tikzpicture}[>=latex',join=bevel,]
\pgfsetlinewidth{1bp}
%%
\pgfsetcolor{black}
  % Edge: i -> map
  \draw [->] (54bp,18bp) .. controls (73bp,18bp) and (99bp,18bp)  .. (136bp,18bp);
  \draw (95bp,26bp) node {$\epsilon/SeqStart_k (pos)$};
  % Edge: map -> j
  \draw [->] (320bp,18bp) .. controls (342bp,18bp) and (364bp,18bp)  .. (392bp,18bp);
  \draw (356bp,26bp) node {$\epsilon/SeqEnd_k (pos)$};
  % Node: i
\begin{scope}
  \pgfsetstrokecolor{black}
  \draw (27bp,18bp) node {$i$};
\end{scope}
  % Node: map
\begin{scope}
  \pgfsetstrokecolor{black}
  \draw (228bp,18bp) node {concat (map doTNFA lst)};
\end{scope}
  % Node: j
\begin{scope}
  \pgfsetstrokecolor{black}
  \draw (419bp,18bp) node {$i+1$};
\end{scope}
%
\end{tikzpicture}
% End of code


      \end{flushleft}
    \item 
      Альтернатива : Alt(a,b) \
      \begin{flushleft}
        \enlargethispage{100cm}
% Start of code
% \begin{tikzpicture}[anchor=mid,>=latex',join=bevel,]
\begin{tikzpicture}[>=latex',join=bevel,]
\pgfsetlinewidth{1bp}
%%
\pgfsetcolor{black}
  % Edge: doA -> j_2
  \draw [->] (228bp,64bp) .. controls (251bp,61bp) and (278bp,56bp)  .. (310bp,51bp);
  \draw (269bp,68bp) node {$\epsilon/Alt1End_k(pos)$};
  % Edge: doB -> j_2
  \draw [->] (228bp,22bp) .. controls (248bp,24bp) and (271bp,28bp)  .. (292bp,32bp) .. controls (295bp,32bp) and (297bp,33bp)  .. (310bp,36bp);
  \draw (269bp,40bp) node {$\epsilon/Alt2End_k(pos)$};
  % Edge: i_2 -> doB
  \draw [->] (54bp,34bp) .. controls (60bp,32bp) and (66bp,30bp)  .. (72bp,29bp) .. controls (90bp,26bp) and (109bp,23bp)  .. (136bp,20bp);
  \draw (95bp,37bp) node {$\epsilon/Alt2Start_k(pos)$};
  % Edge: i_2 -> doA
  \draw [->] (54bp,49bp) .. controls (74bp,53bp) and (101bp,58bp)  .. (136bp,64bp);
  \draw (95bp,67bp) node {$\epsilon/Alt1Start_k(pos)$};
  % Node: doB
\begin{scope}
  \pgfsetstrokecolor{black}
  \draw (182bp,18bp) node {(doTNFA b)};
\end{scope}
  % Node: j_2
\begin{scope}
  \pgfsetstrokecolor{black}
  \draw (337bp,46bp) node {$i+1$};
\end{scope}
  % Node: doA
\begin{scope}
  \pgfsetstrokecolor{black}
  \draw (182bp,72bp) node {(doTNFA a)};
\end{scope}
  % Node: i_2
\begin{scope}
  \pgfsetstrokecolor{black}
  \draw (27bp,44bp) node {$i$};
\end{scope}
%
\end{tikzpicture}
% End of code


      \end{flushleft}
    \item 
      Замыкание : Cls(a) \
      \begin{flushleft}
        \enlargethispage{100cm}
% Start of code
% \begin{tikzpicture}[anchor=mid,>=latex',join=bevel,]
\begin{tikzpicture}[>=latex',join=bevel,]
\pgfsetlinewidth{1bp}
%%
\pgfsetcolor{black}
  % Edge: e_2 -> e_1
  \draw [->] (321bp,28bp) .. controls (309bp,20bp) and (292bp,11bp)  .. (276bp,7bp) .. controls (253bp,0bp) and (246bp,0bp)  .. (222bp,7bp) .. controls (209bp,10bp) and (197bp,16bp)  .. (177bp,28bp);
  \draw (249bp,15bp) node {$\epsilon/ \omega$};
  % Edge: i -> e_1
  \draw [->] (54bp,46bp) .. controls (72bp,46bp) and (96bp,46bp)  .. (126bp,46bp);
  \draw (90bp,54bp) node {$\epsilon/ClsStart_k(pos)$};
  % Edge: e_1 -> e_2
  \draw [->] (180bp,46bp) .. controls (214bp,46bp) and (270bp,46bp)  .. (318bp,46bp);
  \draw (249bp,54bp) node {$\epsilon/ \omega$};
  % Edge: e_2 -> j_3
  \draw [->] (372bp,46bp) .. controls (390bp,46bp) and (414bp,46bp)  .. (444bp,46bp);
  \draw (408bp,54bp) node {$\epsilon/ClsEnd_k(pos)$};
  % Edge: doA_2 -> e_2
  \draw [->] (276bp,84bp) .. controls (286bp,79bp) and (298bp,72bp)  .. (318bp,61bp);
  \draw (297bp,81bp) node {$\epsilon/ \omega$};
  % Edge: e_1 -> doA_2
  \draw [->] (180bp,61bp) .. controls (190bp,66bp) and (202bp,73bp)  .. (222bp,84bp);
  \draw (201bp,80bp) node {$\epsilon/ \omega$};
  % Node: i
\begin{scope}
  \pgfsetstrokecolor{black}
  \draw (27bp,46bp) node {$i$};
\end{scope}
  % Node: e_2
\begin{scope}
  \pgfsetstrokecolor{black}
  \draw (345bp,46bp) node {$i+2$};
\end{scope}
  % Node: e_1
\begin{scope}
  \pgfsetstrokecolor{black}
  \draw (153bp,46bp) node {$i+1$};
\end{scope}
  % Node: j_3
\begin{scope}
  \pgfsetstrokecolor{black}
  \draw (471bp,46bp) node {$i+3$};
\end{scope}
  % Node: doA_2
\begin{scope}
  \pgfsetstrokecolor{black}
  \draw (249bp,99bp) node {(doTNFA a)};
\end{scope}
%
\end{tikzpicture}
% End of code


      \end{flushleft}
  \end{itemize}

\clearpage

После работы построенного таким образом автомата в $globalTagsList$ будет лежать структура, описывающая проесс распознования строк. Она будет являться привильной скобочной структурой(если считать, что метки начала и конца конструкции образуют скобчную пару). 


\section{Построение дерева вывода}

Для построения дерева вывода строки, поданной на вход автомату, можно воспользоваться деей, описанной выше. Для этого потребуется только изменить фукции меток и глобальную стуктуру данных. 

Необходимо преобразовать $globalTagsList$ в стек. Теперь в нём будут лежать не только метки, но и деревья вывода. Фкнкции $Start$ по прежнему будут класть в $globalTagsList$ метку о начале соответствующей конструкции, а $End$ функции будут снимать все элементы до соответствующей метки, строть из них очередной узел, соответствующий распознанной конструкции и класть его обратно.

Таким образом, в конце работы автомата, если строка принята, то в $globalTagsList$ будет лежать дерево вывода этой строки.

\section{Преобразование TNFA в TDFA}

Общая идея преобразования заключается в том, что можно расширить понятие состояния таким образом, чтобы стал возможен перенос в него функций с рёбер. Часть функций будут выполняться на входе в состояние, часть на выходе. 



\section{Пример}

  \begin{flushleft}
        \enlargethispage{100cm}
% Start of code
% \begin{tikzpicture}[anchor=mid,>=latex',join=bevel,]
\begin{tikzpicture}[>=latex',join=bevel,]
\pgfsetlinewidth{1bp}
%%
\pgfsetcolor{black}
  % Edge: 12 -> 13
  \draw [->] (64bp,75bp) .. controls (68bp,75bp) and (73bp,75bp)  .. (87bp,75bp);
  \draw (75bp,83bp) node {$t/\omega$};
  % Edge: 2 -> 3
  \draw [->] (275bp,7bp) .. controls (279bp,7bp) and (284bp,7bp)  .. (298bp,7bp);
  \draw (287bp,15bp) node {$\epsilon/\omega$};
  % Edge: 8 -> 9
  \draw [->] (385bp,75bp) .. controls (355bp,75bp) and (241bp,75bp)  .. (188bp,75bp);
  \draw (287bp,83bp) node {$\epsilon/\omega$};
  % Edge: 4 -> 6
  \draw [->] (349bp,7bp) .. controls (356bp,7bp) and (366bp,7bp)  .. (385bp,7bp);
  \draw (367bp,15bp) node {$\epsilon/SE_1$};
  % Edge: 10 -> 9
  \draw [->] (138bp,75bp) .. controls (145bp,75bp) and (155bp,75bp)  .. (174bp,75bp);
  \draw (156bp,83bp) node {$\epsilon/CS_1$};
  % Edge: 13 -> 10
  \draw [->] (101bp,75bp) .. controls (105bp,75bp) and (110bp,75bp)  .. (124bp,75bp);
  \draw (112bp,83bp) node {$\epsilon/\omega$};
  % Edge: 9 -> 8
  \draw [->] (188bp,77bp) .. controls (206bp,83bp) and (257bp,97bp)  .. (300bp,93bp) .. controls (326bp,90bp) and (357bp,83bp)  .. (385bp,77bp);
  \draw (287bp,101bp) node {$\epsilon/\omega$};
  % Edge: 8 -> 11
  \draw [->] (399bp,75bp) .. controls (406bp,75bp) and (416bp,75bp)  .. (435bp,75bp);
  \draw (417bp,83bp) node {$\epsilon/CE_1$};
  % Edge: 14 -> 12
  \draw [->] (14bp,75bp) .. controls (21bp,75bp) and (31bp,75bp)  .. (50bp,75bp);
  \draw (32bp,83bp) node {$\epsilon/SS_2$};
  % Edge: 6 -> 8
  \draw [->] (392bp,14bp) .. controls (392bp,24bp) and (392bp,43bp)  .. (392bp,68bp);
  \draw (412bp,41bp) node {$\epsilon/\omega$};
  % Edge: 11 -> 15
  \draw [->] (449bp,75bp) .. controls (456bp,75bp) and (466bp,75bp)  .. (485bp,75bp);
  \draw (467bp,83bp) node {$\epsilon/SE_2$};
  % Edge: 1 -> 2
  \draw [->] (238bp,7bp) .. controls (242bp,7bp) and (247bp,7bp)  .. (261bp,7bp);
  \draw (250bp,15bp) node {$+/\omega$};
  % Edge: 5 -> 1
  \draw [->] (188bp,7bp) .. controls (195bp,7bp) and (205bp,7bp)  .. (224bp,7bp);
  \draw (206bp,15bp) node {$\epsilon/SS_1$};
  % Edge: 3 -> 4
  \draw [->] (312bp,7bp) .. controls (316bp,7bp) and (321bp,7bp)  .. (335bp,7bp);
  \draw (324bp,15bp) node {$t/\omega$};
  % Edge: 9 -> 5
  \draw [->] (181bp,68bp) .. controls (181bp,58bp) and (181bp,39bp)  .. (181bp,14bp);
  \draw (199bp,41bp) node {$\epsilon/\omega$};
  % Node: 11
\begin{scope}
  \pgfsetstrokecolor{black}
  \draw (442bp,75bp) ellipse (7bp and 7bp);
  \draw (442bp,75bp) node {11};
\end{scope}
  % Node: 10
\begin{scope}
  \pgfsetstrokecolor{black}
  \draw (131bp,75bp) ellipse (7bp and 7bp);
  \draw (131bp,75bp) node {10};
\end{scope}
  % Node: 13
\begin{scope}
  \pgfsetstrokecolor{black}
  \draw (94bp,75bp) ellipse (7bp and 7bp);
  \draw (94bp,75bp) node {13};
\end{scope}
  % Node: 12
\begin{scope}
  \pgfsetstrokecolor{black}
  \draw (57bp,75bp) ellipse (7bp and 7bp);
  \draw (57bp,75bp) node {12};
\end{scope}
  % Node: 15
\begin{scope}
  \pgfsetstrokecolor{black}
  \draw (492bp,75bp) ellipse (7bp and 7bp);
  \draw (492bp,75bp) node {15};
\end{scope}
  % Node: 14
\begin{scope}
  \pgfsetstrokecolor{black}
  \draw (7bp,75bp) ellipse (7bp and 7bp);
  \draw (7bp,75bp) node {14};
\end{scope}
  % Node: 1
\begin{scope}
  \pgfsetstrokecolor{black}
  \draw (231bp,7bp) ellipse (7bp and 7bp);
  \draw (231bp,7bp) node {1};
\end{scope}
  % Node: 3
\begin{scope}
  \pgfsetstrokecolor{black}
  \draw (305bp,7bp) ellipse (7bp and 7bp);
  \draw (305bp,7bp) node {3};
\end{scope}
  % Node: 2
\begin{scope}
  \pgfsetstrokecolor{black}
  \draw (268bp,7bp) ellipse (7bp and 7bp);
  \draw (268bp,7bp) node {2};
\end{scope}
  % Node: 5
\begin{scope}
  \pgfsetstrokecolor{black}
  \draw (181bp,7bp) ellipse (7bp and 7bp);
  \draw (181bp,7bp) node {5};
\end{scope}
  % Node: 4
\begin{scope}
  \pgfsetstrokecolor{black}
  \draw (342bp,7bp) ellipse (7bp and 7bp);
  \draw (342bp,7bp) node {4};
\end{scope}
  % Node: 6
\begin{scope}
  \pgfsetstrokecolor{black}
  \draw (392bp,7bp) ellipse (7bp and 7bp);
  \draw (392bp,7bp) node {6};
\end{scope}
  % Node: 9
\begin{scope}
  \pgfsetstrokecolor{black}
  \draw (181bp,75bp) ellipse (7bp and 7bp);
  \draw (181bp,75bp) node {9};
\end{scope}
  % Node: 8
\begin{scope}
  \pgfsetstrokecolor{black}
  \draw (392bp,75bp) ellipse (7bp and 7bp);
  \draw (392bp,75bp) node {8};
\end{scope}
%
\end{tikzpicture}
% End of code


  \end{flushleft}

\clearpage

  \begin{flushleft}
        \enlargethispage{100cm}
% Start of code
% \begin{tikzpicture}[anchor=mid,>=latex',join=bevel,]
\begin{tikzpicture}[>=latex',join=bevel,]
\pgfsetlinewidth{1bp}
%%
\begin{scope}
  \pgfsetstrokecolor{black}
  \draw (8bp,107bp) -- (8bp,158bp) -- (98bp,158bp) -- (98bp,107bp) -- cycle;
\end{scope}
\begin{scope}
  \pgfsetstrokecolor{black}
  \draw (124bp,67bp) -- (124bp,158bp) -- (466bp,158bp) -- (466bp,67bp) -- cycle;
\end{scope}
\begin{scope}
  \pgfsetstrokecolor{black}
  \draw (236bp,8bp) -- (236bp,59bp) -- (342bp,59bp) -- (342bp,8bp) -- cycle;
\end{scope}
  \pgfsetcolor{black}
  % Edge: 10 -> 9
  \draw [->] (202bp,133bp) .. controls (210bp,133bp) and (223bp,133bp)  .. (244bp,133bp);
  \draw (223bp,141bp) node {$\epsilon/CS_1$};
  % Edge: 8 -> 9
  \draw [->] (320bp,133bp) .. controls (308bp,133bp) and (284bp,133bp)  .. (258bp,133bp);
  \draw (289bp,141bp) node {$\epsilon/\omega$};
  % Edge: 4 -> 6
  \draw [->] (376bp,87bp) .. controls (368bp,86bp) and (355bp,86bp)  .. (334bp,84bp);
  \draw (355bp,96bp) node {$\epsilon/SE_1$};
  % Edge: 2 -> 3
  \draw [->] (258bp,39bp) .. controls (270bp,39bp) and (294bp,39bp)  .. (320bp,39bp);
  \draw (289bp,48bp) node {$\epsilon/\omega$};
  % Edge: 13 -> 10
  \draw [->] (146bp,133bp) .. controls (154bp,133bp) and (167bp,133bp)  .. (188bp,133bp);
  \draw (167bp,141bp) node {$\epsilon/\omega$};
  % Edge: 9 -> 8
  \draw [->] (253bp,126bp) .. controls (256bp,117bp) and (264bp,101bp)  .. (276bp,94bp) .. controls (287bp,89bp) and (292bp,89bp)  .. (302bp,94bp) .. controls (311bp,99bp) and (317bp,108bp)  .. (325bp,126bp);
  \draw (289bp,106bp) node {$\epsilon/\omega$};
  % Edge: 8 -> 11
  \draw [->] (334bp,133bp) .. controls (342bp,133bp) and (355bp,133bp)  .. (376bp,133bp);
  \draw (355bp,141bp) node {$\epsilon/CE_1$};
  % Edge: 14 -> 12
  \draw [->] (30bp,133bp) .. controls (39bp,133bp) and (54bp,133bp)  .. (76bp,133bp);
  \draw (53bp,141bp) node {$\epsilon/SS_2$};
  % Edge: 3 -> 4
  \draw [->] (334bp,45bp) .. controls (341bp,50bp) and (350bp,58bp)  .. (358bp,65bp) .. controls (361bp,68bp) and (365bp,71bp)  .. (376bp,81bp);
  \draw (355bp,74bp) node {$t/\omega$};
  % Edge: 12 -> 13
  \draw [->] (90bp,133bp) .. controls (98bp,133bp) and (111bp,133bp)  .. (132bp,133bp);
  \draw (111bp,141bp) node {$t/\omega$};
  % Edge: 11 -> 15
  \draw [->] (390bp,133bp) .. controls (400bp,133bp) and (419bp,133bp)  .. (444bp,133bp);
  \draw (417bp,141bp) node {$\epsilon/SE_2$};
  % Edge: 1 -> 2
  \draw [->] (202bp,81bp) .. controls (207bp,77bp) and (214bp,70bp)  .. (220bp,65bp) .. controls (225bp,61bp) and (231bp,56bp)  .. (244bp,45bp);
  \draw (223bp,77bp) node {$+/\omega$};
  % Edge: 5 -> 1
  \draw [->] (244bp,84bp) .. controls (236bp,85bp) and (223bp,85bp)  .. (202bp,87bp);
  \draw (223bp,96bp) node {$\epsilon/SS_1$};
  % Edge: 6 -> 8
  \draw [->] (327bp,116bp) .. controls (327bp,107bp) and (327bp,97bp)  .. (327bp,126bp);
  \draw (345bp,108bp) node {$\epsilon/\omega$};
  % Edge: 9 -> 5
  \draw [->] (251bp,100bp) .. controls (251bp,109bp) and (251bp,119bp)  .. (251bp,90bp);
  \draw (269bp,108bp) node {$\epsilon/\omega$};
  % Node: 11
\begin{scope}
  \pgfsetstrokecolor{black}
  \draw (383bp,133bp) node {13};
\end{scope}
  % Node: 10
\begin{scope}
  \pgfsetstrokecolor{black}
  \draw (195bp,133bp) node {4};
\end{scope}
  % Node: 13
\begin{scope}
  \pgfsetstrokecolor{black}
  \draw (139bp,133bp) node {3};
\end{scope}
  % Node: 12
\begin{scope}
  \pgfsetstrokecolor{black}
  \draw (83bp,133bp) node {2};
\end{scope}
  % Node: 15
\begin{scope}
  \pgfsetstrokecolor{black}
  \draw (451bp,133bp) node {14};
\end{scope}
  % Node: 14
\begin{scope}
  \pgfsetstrokecolor{black}
  \draw (23bp,133bp) node {1};
\end{scope}
  % Node: 1
\begin{scope}
  \pgfsetstrokecolor{black}
  \draw (195bp,88bp) node {7};
\end{scope}
  % Node: 3
\begin{scope}
  \pgfsetstrokecolor{black}
  \draw (327bp,39bp) node {9};
\end{scope}
  % Node: 2
\begin{scope}
  \pgfsetstrokecolor{black}
  \draw (251bp,39bp) node {8};
\end{scope}
  % Node: 5
\begin{scope}
  \pgfsetstrokecolor{black}
  \draw (251bp,83bp) node {6};
\end{scope}
  % Node: 4
\begin{scope}
  \pgfsetstrokecolor{black}
  \draw (383bp,88bp) node {10};
\end{scope}
  % Node: 6
\begin{scope}
  \pgfsetstrokecolor{black}
  \draw (327bp,83bp) node {11};
\end{scope}
  % Node: 9
\begin{scope}
  \pgfsetstrokecolor{black}
  \draw (251bp,133bp) node {5};
\end{scope}
  % Node: 8
\begin{scope}
  \pgfsetstrokecolor{black}
  \draw (327bp,133bp) node {12};
\end{scope}
%
\end{tikzpicture}
% End of code


  \end{flushleft}

  \begin{flushleft}
        \enlargethispage{100cm}
% Start of code
% \begin{tikzpicture}[anchor=mid,>=latex',join=bevel,]
\begin{tikzpicture}[>=latex',join=bevel,]
\pgfsetlinewidth{1bp}
%%
\pgfsetcolor{black}
  % Edge: n1 -> n2
  \draw [->] (136bp,168bp) .. controls (136bp,160bp) and (136bp,150bp)  .. (136bp,131bp);
  % Edge: n3 -> n2
  \draw [->] (224bp,36bp) .. controls (199bp,51bp) and (188bp,51bp)  .. (160bp,73bp);
  % Edge: n2 -> n3
  \draw [->] (244bp,73bp) .. controls (221bp,54bp) and (211bp,53bp)  .. (181bp,36bp);
  % Node: n1
\begin{scope}
  \pgfsetstrokecolor{black}
  \draw (0bp,168bp) -- (0bp,204bp) -- (163bp,204bp) -- (163bp,168bp) -- cycle;
  \draw (49bp,168bp) -- (49bp,204bp);
  \draw (110bp,168bp) -- (110bp,204bp);
  \draw (24bp,186bp) node {start};
  \draw (79bp,186bp) node {14, 12};
  \draw (136bp,186bp) node {finish};
\end{scope}
  % Node: n2
\begin{scope}
  \pgfsetstrokecolor{black}
  \draw (111bp,72bp) -- (111bp,131bp) -- (297bp,131bp) -- (297bp,72bp) -- cycle;
  \draw (160bp,72bp) -- (160bp,131bp);
  \draw (244bp,72bp) -- (244bp,131bp);
  \draw (135bp,113bp) node {start};
  \draw (202bp,113bp) node {1 ,4 ,5 ,6 };
  \draw (202bp,96bp) node { 8, 9, 10 };
  \draw (202bp,79bp) node { 11, 15};
  \draw (270bp,113bp) node {finish};
\end{scope}
  % Node: n3
\begin{scope}
  \pgfsetstrokecolor{black}
  \draw (132bp,0bp) -- (132bp,36bp) -- (277bp,36bp) -- (277bp,0bp) -- cycle;
  \draw (181bp,0bp) -- (181bp,36bp);
  \draw (224bp,0bp) -- (224bp,36bp);
  \draw (156bp,18bp) node {start};
  \draw (202bp,18bp) node {2, 3};
  \draw (250bp,18bp) node {finish};
\end{scope}
%
\end{tikzpicture}
% End of code


  \end{flushleft}

\end{document}
