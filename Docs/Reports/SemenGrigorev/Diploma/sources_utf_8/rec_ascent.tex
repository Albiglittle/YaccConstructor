\subsection{Рекурсивно восходящий алгоритм}


Существует альтернатива табличным анализаторам -- рекурсивно-восходящие  анализаторы. Идея состоит в том, чтобы не использовать стек явно, а заменить его стеком вызова рекурсивных функций и эмулировать поведения автомата вызовом функций. Для этого можно построить функцию соответствующую каждому состоянию и получить детерминированный LR анализ ~\cite{RECURSIVE-ASCENT PARSING}.  

Таким образом - рекурсивно восходящий алгоритм это аналог рекурсивного спуска, но для LR грамматики.  Однако, при такой реализации возникают проблемы с быстрым ростом объёма кода целевых функций~\cite{Jade}. 

Существует подход к решению этой задачи несколько с другой стороны. Показано, что кожно построить систему функций, оперирующие уже не одним состоянием, а множеством состояний, и свести ,таким образом, количество необходимых функций к двум взаимно рекурсивным~\cite{RecursiveAscentParsing}:  
\begin{itemize}
	\item parse  q i =$\{(A\stackrel{}{\rightarrow}a. , i) | A\stackrel{}{\rightarrow} a. \in q\}\bigcup$
  
  \hspace{1,9cm}       $\{(A\stackrel{}{\rightarrow}a.b , k) | i = xj ,(A\stackrel{}{\rightarrow}a.b, k) \in climb$ q x j$  \}
  \bigcup$
  
  \hspace{1,9cm}       $\{(A\stackrel{}{\rightarrow}a.b , k) | B\stackrel{}{\rightarrow}e , (A\stackrel{}{\rightarrow}a.b, k) \in climb$ q B j $\}$
  \item climb q X  i = $\{(A\stackrel{}{\rightarrow}a.Xb , k) | (A\stackrel{}{\rightarrow}aX.b, k)\in parse($goto q X$) i , a\neq e, A\stackrel{}{\rightarrow}a.Xb \in q\}\bigcup$
  
  \hspace{2,5cm}          $\{(A\stackrel{}{\rightarrow}a.b , l) | (C\stackrel{}{\rightarrow}X.c,j)\in parse($goto q X$) i, (A\stackrel{}{\rightarrow}a.b ,l)\in climb$ q C j$\} $
\end{itemize}
 
%Пользуясь таким определением можно построить функции из статьи ~\cite{RECURSIVE-ASCENT PARSING}, однако сейчас интереснее их реализация, согласно определению.

Важно, что при решении нашей задачи нет необходимости генерации функции для каждого состояния. Так как цель - получение недетерминированного разбора, то можно реализовать всего две функции: parse и climb, определённые выше. Это позволит решить проблему объёма кода, возникшую в Jade. Объём кода резко сократится и, более того, будет получен недетерминированный вариант разбора.  

Используя рекурсивно-восходящий алгоритм можно получить эквивалент  алгоритму Томиты. При  использовать описанных выше функции автоматически получается ветвление в момент возникновения неоднозначности (разделение стека в алгоритме Томиты). Это получается благодаря тому, что они принимают на вход состояние и возвращают множество всех возможных состояний. 

Однако возникает проблема - сложность алгоритма, основанного на функциях parse и climb экспоненциальная. Для борьбы с этим в статье~\cite{Non-det-rec-asc} предлагается дополнить функции  механизмом запоминания результатов предыдущих вызовов, чтобы при очередном вызове, в случае, если эта функция уже вызывалась с такими параметрами, то результат возвращался без повторного вычисления. Доказывается~\cite{Non-det-rec-asc}, что в при такой реализации оценки по времени будут порядка О($n^{3}$).

Запоминание результатов предыдущих вычислений и будет аналогом объединения состояний и получения структурированного в виде графа стека (graph-structured stack) в алгоритме Томиты.

Технически этого можно добиться реализовав функцию memoize. На языке F\# её можно   реализовать так:
\begin{verbatim}
      let memoize (f: 'a ->'b) =
           let t = new System.Collections.Generic.Dictionary<'a,'b>() 
           in
           fun a ->    
                  if t.ContainsKey(a)
                  then t.[a]
                  else 
                     let res = f a 
                     in
                     t.Add(a,res);
                     res 
\end{verbatim}

В дальнейшем в качестве функции f будет выступать функция parse и climb.     

Дополнительного ускорения можно добиться заранее вычислив функцию goto. Действительно, построение замыкания (closure), необходимое для вычисления функции goto дорогая операция, а вызов goto происходит при каждом вызове функции climb. Поэтому можно вычислить goto на этапе построения анализатора, и в целевой программе (в функции climb) свести вызов (goto q X) к поиску по ключу.

%С учётом этих дополнений на F\# функции \verb| parse| \ и \verb| climb| \ могут быть реализованы следующим образом:
%\begin{verbatim}
%let rec climb =
%    memoize (fun (states,(symbol,i)) -> 
%    if states = Set.empty
%    then Set.empty
%    else     
%    let gt =  goto (states,symbol)    
%    let new_states = parse (gt,i)   
%    if Set.exists (fun ((state,tree),i)-> 
%                       state.prod_name="S"
%                        &&
%                       state.next_num=None&&i=1)
%                       new_states      
%    then states
%    else    
%    Set.union_all                            
%    [Set.filter (fun items-> 
%                   Set.exists (fun item  -> 
%                                   (exists ((=)(fst items))(nextItem item))
%                                    &&
%                                   (item.item_num <> item.s)
%                               )states)new_states
%     |>(Set.map (fun items->
%                     (map (fun itm -> 
%                               (itm,snd items))
%                          (prevItem (fst items)))))|>union_all
%    ;
%    Set.union_all(
%    union_from_Some[for (item,i) in new_states -> 
%                    if (exists (fun itm -> (getText itm.symb) = x) 
%                               (prevItem item)) 
%                        && 
%                       (item.prod_name<>"S")
%                        &&
%                       (exists (fun itm -> itm.item_num=item.s)(prevItem item))
%                    then Some(climb (states,item.prod_name,i))
%                    else None])
%    ])                
%and parse = 
%    memoize (fun (states,i) ->    
%    union_all
%        [map (fun state -> (state,i))
%             (Set.filter (fun item -> (item.next_num=None))states)
%         ;if (get_next_lex i = m_end) 
%          then empty 
%          else climb(states,mgetText(get_next_lex i),i-1)
%        ])
%\end{verbatim}
