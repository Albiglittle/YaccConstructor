\documentclass{diploma}
\usepackage{cmap} % for serchable pdf's
\usepackage[T2A]{fontenc} 
\usepackage[utf8]{inputenc}
\usepackage[english,russian]{babel}
\usepackage{indentfirst}
\usepackage[colorlinks=true,urlcolor=black,linkcolor=black,filecolor=black,citecolor=black,unicode,pdftex]{hyperref}
\usepackage{graphicx}
\usepackage{caption}
\usepackage{pdfpages}
\usepackage{amsmath}
\usepackage{dot2texi}
\usepackage{tikz}
\usetikzlibrary{shapes,arrows}
\usepackage{listings}
%\usepackage[titletoc]{appendix} 
%\usepackage{citehack}
\renewcommand{\baselinestretch}{1.5}


% to economize paper (for printing) uncomment next 5 lines
%\textwidth=190mm
%\textheight=250mm
%\topmargin=-20mm
%\oddsidemargin=-15mm
%\evensidemargin=-15mm



\begin{document}
\sloppy

\title{Генератор синтаксических анализаторов для неоднозначных контекстно-свободных грамматик}
\author{Григорьева Семёна Вячеславовича}
\university{Санкт-Петербургский Государственный Университет}
\facility{Математико-механический факультет}
\group{461}
\position{студента}
\chair{Кафедра системного программирования}
\leaderPosition{к.ф.-м.н.}
\leader{А.С. Лукичев}
\criticPosition{д.ф.-м.н., проф.}
\critic{Б.К. Мартыненко}
\chairLeaderPosition{д.ф.-м.н., проф.}
\chairLeader{А.Н. Терехов }
\city{Санкт-Петербург}
\yr{2010}

\maketitle
\includepdf{SemenDiplomaTitle_en.pdf}
\setcounter{page}{2}
\tableofcontents
\newpage


\section{��������}	

������ �������������������  �������������  �������� ��������� ������ ���������� � ����������� �������������� ������������.

��� ����������� ����� ������ (� �������� � ����������) ������ ����������� ����������-��������� ����������. ���������� ����������� ������������� ��� ������������, ������� ���������� � ������������, ����� �������� ����� ����������, �� ��� ���� ��� ��������� ���� ��������������~\cite{CurrentParsTechn}. ������� ���������� ���� ������ �������� � ������� ������������� ����������-��������� ����������.

��� ��������� ���������� ��������� ����� ���������� ����������� ����� ������������� ����������. ������,��������, ��������� ������ ������� �������� � ��������� �������� ���������� � ����������~\cite{CurrentParsTechn}, ������� ���������� ��������� �������.
��� ������� �������� ���������� �������. 

��� ������� ������� ���� ����� ������������ ������������ GLR ���������� � ��������������� ����������� ���������� ������������~\cite{CurrentParsTechn}. �������������, GLR-�������� ��������� ��������������� � ���������� �� ������ ���������. �� �����  ������� ������������ ���������� ���������� �����, ������� ������ �������. ������������ ��� ���������� � ��������������. 
      
������� ������������ GLR-��������� �������� ��������� ������������� ���������. ����������, ����������� �� �������������  ���������� � ������� ������� ���������, � ���������� ������� ������ �� ������������ ������, � ��������� �������� - ���, ������� ����� ���������, ��������� ����������� �������, � ����� ,��� ������� � ����� ������������ ���������� ���������, ������� ���� ��� ��� ����������� ������ ������� ������/��������.   

������ ��������� GLR ����� ������������� ��� ������������ ���������� ������ LR-������������. ��� ���� ������ ����� ����������� ���������� ���������� �������, �������������� ������������� ������ ����� �� ������������ � ��������������, ��� ��������� ������� � ������� ������������ ������ � ������ ������ LR-�����������, ���� � ������� �� �������� �������� ������������ ����������.

���������, ��� ������ �������� ����� �������� ����� ���� ����������� � ���� ���� �������-����������� ������� (����������-���������� ��������, recursive ascent). ��� ���� ������������ ����� ���������� ������������ ������� ��� ��������� � ����� �� �������, � �������� ���������� ����� ���� ����������� ��� ����������� ���������� �������.

����� ��������, ��� �� ������������������ ����� ����������, ������� ��������� "�����������" \ ��� LR-������������, ������������� ��� ��������. �� ����������� ���� � ����������� ������������������/����� ����������� ������ GLR-�������� �������� �������� ���������������.

������� �������� ����������� ����������� ����������, ��������� � ��������� ����� ���������������� �������� ����������� ���������� ����� ������-�����. ������� ���������� ������ �������� � ������������ ����������-���������� ������������. 

��� ������ � ������������ ������������ ������� �������� ��������� ��������� � �������� �������� �� ����������. ��� ��������� �������������� ���������� � ���������. � ������, ���� ������� ���������� ���� �����-���� ������� �������������, �������� � ����� �������� ����������� EBNF, �� ���������� ������������� � ���������� "���������" \ ��������������. ��� �������������� ������ "���������" \ ��������� ������� � ������� ������� ����������. ����� ��������������  ������� �������������� �������� � ��������� ����������. ������� �������� ����������������� �������� ���������, ����������� �������� ��� �������������� �������������� ����������.
\clearpage
\section{���������� ������}

� ���� �������������� ��������� ���������� � ���������� �������������� ������������:
\begin{itemize}
	\item ����������� ������ � �������������� ������������. 
	\item ����������� ������������ �������� ���� ���������, ������������� � ���������� ��������� ������. ��� ������� ���������� � EBNF.
	\item ��������� ���������� ���������.
\end{itemize}

���� ������ ������:  ����������� �������� �����������, ����������� �������������� ���� �������������.
\clearpage
\section{Основные определения}

Определим ряд понятий и введём некоторые обозначения, необходимых для дальнейшего изложения.
\\
\\
	{\bfseries Определение 1.} \textit{Конструкции регулярнх выражений}: будем говорить, что грамматика (правило) содержит конструкции регулярных выражений, если в записи правых частей правил используются элементы синтаксиса регулярных выражений (альтернатива, замыкание).
\\
	{\bfseries Определение 2.} \textit{Раскрытие конструкций регулярных выражений}: будем называть раскрытием конструкций регулярных выражений такое преобразование грамматики (правила), при котором исходная грамматика заменяется на эквивалентную, не содержащую конструкций регулярных выражений.
\\
	{\bfseries Определение 3.} \textit{Дерево вывода строки в EBNF-грамматике}: упорядоченное помеченное дерево $D$ называется деревом вывода в EBNF-грамматике $G(S)=(N,T,P,S)$, если выполнены следующие условия:

\begin{enumerate}
	\item корень дерева $D$ помечен $S$;
	\item каждый лист помечен либо $a \in T$, либо $\varepsilon$;
	\item каждая внутренняя вершина помечена нетерминалом;
	\item если $N$ -- нетерминал, которым помечена внутренняя вершина и $X_1,...,X_n$ - метки ее прямых потомков в указанном порядке, то существует правило $N \rightarrow Y_1...Y_n \in P$ такое, что строка $X_1...X_n$ пораждается регулярным выражением $Y_1...Y_n$.
\end{enumerate}
	{\bfseries Определение 4.} \textit{Непосредственная поддержка EBNF-грамматик}: будем говорить, что инструмент непосредственно поддерживает EBNF-грамматики, если он работает с ними без раскрытия конструкций регулярных выражений.
\\
 	{\bfseries Определение 5.} \textit{Побочный эффект}: будем говорить, что функция или атрибут обладают побочным эффектом, если в процессе их вычислений возможно читать и модифицировать значения глобальных переменных, осуществлять операции ввода/вывода, реагировать на исключительные ситуации, вызывать их обработчики.



Для примеров псевдокода, приводимых далее, будем использовать синтаксические соглашения, принятые в языке программирования F\#.

%\begin{itemize}

	%\item GLR: Generalized LR Parsing~\cite{CurrentParsTechn}

	%%\item ДКА: детерминированный конечный автомат. Такой автомат, в котором для каждой последовательности входных символов существует лишь одно состояние, в которое автомат может перейти из текущего~\cite{DrgBook}.
	
	%%\item НКА: недетерминированный конечный автомат~\cite{DrgBook}.
	
	%\item Замыкание: q* = q$ \bigcup \{B\rightarrow.c | A \rightarrow a.Bb \in $q*$\} \bigcup \{x\stackrel{}{\rightarrow}.x | A\stackrel{}{\rightarrow} a.xb \in $q*$\}$~\cite{DrgBook}
	
	%%\item LR-ситуация: продукция с точкой в некоторой позиции правой части~\cite{DrgBook}.

%\end{itemize}

\clearpage
\section{�����.}

����� ������ ���������� ����������� ��� ������������ ���������� ��������� ��������� �� ��������� .NET � ���������� ��� �� �������������� ����� ���������������� F\#. ������������� ����� ���������� ���� ���������� ����. 
         
������������������ ������������ �������� ��������� ������ ���������� � ����������� ������� � ����� ������������ ���������� \cite{Diploma}. ������������ � ��������������� ����������� ���������� YARD ������������� �������� ����� �����������.  ������� ���������� ������� ������ ������������ ������ ���������� ���������� ������� � ������������ ������������. 
          
���������������� ���������� ������� �������� �������� ������� ������������ ���������� ��������� ����������. ������� ����� ����������� �����������, ���������� �� ���� ���������. ����� ������ ���������� ���������, ��� ��� ���������� ��������� �����������: �������� ������ (GLR-��������), �������� ���� (Early), ����������-���������� ��������. ���-�� ������� �������� �������������� ����������, ����������� ���  ���������� �����������. 
          
��� ��� ���� ���������������� F\# �������� ��������������, �� ������ ������� ������������ ����������-���������� ��������, ��� ��� GLR-�������� � �������� ���� ���� �����������. � ����� � ���� ������� ������������ ���������� Jade - ��������� ���������� ���������� ��������. 
       
� ��������� ����� ��� GLR ������������ �� ��������� .NET. �� ���������� ������ ����������� ���������� �� GLR-���������.
\begin{itemize}
\item
 ASF+SDF \cite{ASF+SDF} (Algebraic Specification Formalism + Syntax Definition Formalism)
- ��������� � �������� �������������, �� ���������� ������� �������
������. �������� SGLR-������������ (Scannerless, Generalized-LR).
\item
 Bison \cite{Bison} - �������� ����������� YACC. ��� ����������, ���������
��� ������������� YACC, ����� �������� � � Bison. �������� �����
�� ����� ���������� � ����������� "��������" YACC. ��� ���������
��������������� ����� ���������� GLR-�������� (�� ��������� LALR).
\item
 Elkhound \cite{Elkhound} - ��������������� ��� ������� � ������� GLR-����������,
��������� � ������������ ������ (���), ��� �� ����� �������� ����������
"������" ������� ������ (��������, �� �� ������������ �����������
����������� ����� ������-�����).
\end{itemize}

� ������ \cite{Diploma} ������� ��������� ������ ���� ������������. ������� ����, ����� �������, ��� �� ������� ������ ��� �����������, ��������� ���������������� ����������� ������������������� ������������� ��������.

��� �� ��� ����� ������������ ����� ���������� ��� Jade, ��� ��� �� �������� ����������� ����������-���������� ��������.
       
Jade ��� ��������� ����������-���������� LALR(1) �������� � ������� ������ �. ��� ��������� �������� ���������� � ������ \cite{Jade}. ��� ��� ���������� ��������� �������� ������ ���� �������� �������. ��� ��� ��� ���������� ������������������ ������� ���������� ������������ ��������� ��� ������� ���������, �� ����� ���� ������ �����, � ������ ���������� ������ � ����������. ��� ��� ����� Java, �� ��������, ���������� � ������\cite{Jade}, ����� ���� ���������� �������� 4 ���������. � Jade  ��� �������� �������� ���� �������� ���������� ���������(������� ���������), ��� �������� ����������, ����������� ���������������� ���������. (��������� �� ���� ����� �������� � ������ \cite{Jade}). 


\clearpage
\section{Реализация}
 
%\section{Описание решения}

%Необходимость поддержки неоднозначных грамматик предопределила выбор GLR-алгоритма в качестве анализатора.
Для реализации выбран рекурсивно-восходящий алгоритм, модернизированный для работы с расширенными контекстно-свободными грамматиками без их преобразования.

В качестве фронтенда, обладающего мощным и удобным языком спецификации трансляции, был выбран инструмент YARD~\cite{Diploma}. Он позволяет получить дерево разбора грамматики, заданной пользователем, которое используется в дальнейшем.%Вопрос к Якову Александровичу - как правильно писать? будет ли ребрендинг?

Выбранный подход к вычислению атрибутов -- интерпретация дерева вывода. Основная идея заключается в том, что генератор строит набор функций, каждая функция соответствует одному правилу грамматики. После построения дерева вывода, оно обходится снизу вверх и в каждом узле вычисляется соответствующая ему функция. Функции вызываются по рефлексии, что показывает возможность в будущем сделать инструмент более гибким благодаря возможности динамически (в процессе работы анализатора) изменять функции вычисления атрибутов (это возможно благодаря возможностям динамической перекомпиляции, предоставляемыми языком реализации).

Инструмент реализован на платформе .NET~\cite{.NET}, на функциональном языке F\#~\cite{FS}.

Более подробно детали реализации описаны ниже.



\subsection{Алгоритм анализа. Основные функции}

Рекурсивно-восходящий алгоритм позволяет с помощью набора взаимно-рекурсивных функций эмулировать поведение LR-автомата. Его можно рассматривать, как аналог рекурсивного спуска, но для LR грамматики. Действительно, стек автомата естественным образом заменяется на стек вызова функций, а вызовы функций заменяют переходы автомата.

При таком подходе возникает проблема с объёмом кода целевого инструмента. Связана она с тем, что количество функций, которые необходимо построить, сравнимо с размером таблицы переходов LR-автомата, которые на практике бывают очень большими.

Для решения этой проблемы можно воспользоваться подходом, предложенным в работе~\cite{RecursiveAscentParsing}. Основная идея предложенного подхода -- реализовать всего две взаимно-рекурсивные функции, но оперирующие при этом уже не одним состоянием, а множеством состояний. Важно, что при такой реализации можно получить недетерминированный анализ. При этом ветвление реализуется, как ветвление в одной из функций, а механизм для переиспользования результатов вычисления, слияние, можно реализовать, как запоминание результатов вычисления функций. Для этого можно воспользоваться такой конструкции функционального программирования, как замыкание. 

В основе этого алгоритма лежат две взаимно-рекурсивные функции $parse$ и $climb$, которые можно определить следующим образом:
\begin{itemize}
	\item parse  q i =$\{(A\stackrel{}{\rightarrow}a. , i) | A\stackrel{}{\rightarrow} a. \in q\}\bigcup$
  
  \hspace{1,9cm}       $\{(A\stackrel{}{\rightarrow}a.b , k) | i = xj ,(A\stackrel{}{\rightarrow}a.b, k) \in climb$ q x j$  \}
  \bigcup$
  
  \hspace{1,9cm}       $\{(A\stackrel{}{\rightarrow}a.b , k) | B\stackrel{}{\rightarrow}e , (A\stackrel{}{\rightarrow}a.b, k) \in climb$ q B j $\}$
  \item climb q X  i = $\{(A\stackrel{}{\rightarrow}a.Xb , k) | (A\stackrel{}{\rightarrow}aX.b, k)\in parse($goto q X$) i , a\neq e, A\stackrel{}{\rightarrow}a.Xb \in q\}\bigcup$
  
  \hspace{2,5cm}          $\{(A\stackrel{}{\rightarrow}a.b , l) | (C\stackrel{}{\rightarrow}X.c,j)\in parse($goto q X$) i, (A\stackrel{}{\rightarrow}a.b ,l)\in climb$ q C j$\} $
\end{itemize}

Где:
\begin{itemize}
  \item $G=(V_T,V_N,P,S)$: контекстно-свободная грамматика;
  \item $V = V_T \cup V_N$;  
  \item $x,y \in V_T$
  \item $i,j \in V_T^*$
  \item $X,Y \in V$
  \item $A,B,C \in V_N$
  \item $a,b,c \in V^*$ 
  \item $q$: LR-состояние (core) 
	\item $goto \ q \ X $ = $\{A\stackrel{}{\rightarrow}aX.b | A\rightarrow a.Xb \in $q*$\}, $
  \item \textit{q*}: замыкание. q* $= q \bigcup \{B\rightarrow.c | A \rightarrow a.Bb \in $q*$\} \bigcup \{x\stackrel{}{\rightarrow}.x | A\stackrel{}{\rightarrow} a.xb \in $q*$\}$~\cite{DrgBook}
\end{itemize}

Далее, на основе этого алгоритма надо получить алгоритм, реализующий:
\begin{enumerate}
 	\item непосредственную поддержку EBNF-грамматик;
 	\item построение леса вывода.
 \end{enumerate} 

Для того в него необходимо внести некоторые изменения, описанные далее.

\subsubsection{Поддержка расширенных контекстно-свободных грамматик}

Поддержка регулярных выражений в правых частях правил (EBNF-грамматики) получается естественным образом. Для этого правая часть правила представляется как детерминированный конечный автомат. LR-ситуация в таком случае может быть представлена парой: правило (нетерминал+КА) и номер состояния (соответствует позиции маркера в классическом определении).

Действительно, в правой части правила всегда находится регулярное выражение. В простом случае, когда это последовательность терминалов и нетерминалов, позиция маркера из классического определения LR-ситуации тривиальным образом соответствует состоянию конечного автомата, построенного по этой последовательности, а перемещение маркера -- переход КА из одного состояния в другое. В общем случае по  регулярному выражению из правой части правила по алгоритму Томпсона~\cite{DrgBook} строится недетерминированный конечный автомат (НКА). Заметим, что в полученном НКА много $\varepsilon$-переходов. Например, каждая альтернатива вносит два дополнительных состояния и 4 $\varepsilon$-перехода. Чтобы уменьшить количество переходов в результирующем LR-автомате можно преобразовать НКА в детерминированный конечный автомат(ДКА). Для этого применим стандартный алгоритм преобразования НКА в ДКА~\cite{DrgBook}. После этого заменим позицию маркера номером состояния полученного ДКА.
 
Таким образом, мы можем строить LR-ситуации для EBNF-грамматик. Для каждой, построенной LR-ситуации, для дальнейшей работы необходимо хранить следующую информацию:
\begin{itemize}
  \item номер правила;
  \item левую часть правила(нетерминал);
  \item номер текущего состояния ДКА;
  \item символ принимаемый ДКА в данном состоянии;
  \item номер состояния ДКА, в которое он перейдёт приняв данный символ;
  \item номер начального состояния ДКА;
  \item номера конечных состояний ДКА;
\end{itemize}

Функция $goto$ для работы с новыми LR-ситуациями основана на стандартном алгоритме вычисления GOTO при LR анализе~\cite{DrgBook}. Его необходимо лишь изменить таким образом, чтобы он мог работать с LR-ситуациями, определённого нами вида. 

\subsubsection{Построение леса вывода}

Для того, чтобы построить лес вывода, необходимо добавить в функции механизм конструирования очередного узла дерева и предусмотреть сохранение леса.

В функции $parse$ необходимо производить конструирование нового листа, в случае, когда происходит чтение очередного символа из входной цепочки.

В функции $climb$ необходимо конструировать новый внутренний узел из поддеревьев, в случае, когда нужно произвести свёртку по текущему правилу, и объединять множества поддеревьев, в случае, если работа с текущем правилом ещё не закончена (сместился маркер в правой части).

Введём следующие обозначения:
\begin{itemize}
  \item $A \rightarrow R$ -- правило грамматики, где $A$ -- нетерминал, $R$ -- ДКА, построенный по регулярному выражению; 
  \item $(A \rightarrow R,i)$ -- LR-ситуация, где $i$ -- состояние ДКА;
  \item $is-final \ R \ i$ -- функция , осуществляющая проверку, что $i$ -- конечное состояние $R$;
  \item $exists \ elem \ set $ -- функция, проверяющая наличие элемента $elem$ в множестве $set$;
  \item $(leaf:a)$, $(A->...)$ -- конструкции дерева разбора;
\end{itemize} 

Тогда сигнатуры функции будут иметь следующий вид:

$parse$ $ q $ $ \{ u | A \rightarrow a.b, u = vx, b \rightarrow v \} \rightarrow (A \rightarrow a.b) x \{узлы синтаксического дерева для b \})$

$climb$ $q$ $X$ $\{ u | A \rightarrow aX.b, u = vx, b \rightarrow v \} (tree $: синтаксическое дерево для X$) \rightarrow (A \rightarrow a.Xb) x \{узлы синтаксического дерева для Xb\}$

А сами функции будут выглядеть так:

\verb|parse q u =|

\ \ \ \ \ \  \verb|if exists (A| $\rightarrow$ \verb|R,i) q| \ $\&$ \ \verb|is-final(R,i)| 
  
\ \ \ \ \ \  \verb|then (A| $\rightarrow$ \verb|R,i;u;[])|

\ \ \ \ \ \  \verb|else|

\ \ \ \ \ \ \ \ \ \verb|if u=av| $\&$ \verb|exists ( A| $\rightarrow$ \verb| R,i) q* | $\&$ \verb| R(i,a)=j| 
     
\ \ \ \ \ \ \ \ \ \verb|then climb q a v (leaf:a)|
     
\ \ \ \ \ \  \verb|else|
     
\ \ \ \ \ \  \verb|if exists (A| $\rightarrow$ \verb|R,0) q*| \ $\&$ \ \verb|is-final(R,0)| 
        
        
\ \ \ \ \ \  \verb|then climb q A v (A| $\rightarrow$ \verb|[])|
        
\verb|climb q X u h = |

  \ \ \ \ \ \ \verb| let (A| $\rightarrow$ \verb|R,j;w;s) = parse (goto q* X) u in|
  
  \ \ \ \ \ \ \verb| if R(i,X)=j| \ $\&$ \ \verb|exists (A|$\rightarrow$ \verb| R,i) q| 
  
  \ \ \ \ \ \ \verb| then (A| $\rightarrow$ \verb| R,i;w;h::s)|
   
  \ \ \ \ \ \ \verb| else climb q A w (A| $\rightarrow$ \verb| h::s)|


\subsection{Вычисление атрибутов}

На практике для задания семантических действий применяются наследуемые (l-атрибутные грамматики) и вычислимые (s-атрибутные грамматики) атрибуты. В рамках данной работы была поставлена задача поддержать работу с s-атрибутными грамматиками.

Далее будут подробнее описаны особенности вычисления атрибутов при непосредственной поддержке EBNF-грамматик и алгоритм вычисления атрибутов.


\subsubsection{Задание семантических атрибутов в YARD}

Грамматика YARD-а~\cite{Diploma} позволяет определять атрибуты для любой части продукции, которая является последовательностью. На практике это означает, что атрибут может быть ассоциирован не только с правилом целиком, а с любой его частью, которая является последовательностью. Например:

\begin{verbatim}
  someRule : val1 = (a {action1} | b {action2}) 
             val2 = c  {someFunc val1 val2};
\end{verbatim}

Здесь альтернатива \verb| ( a |\verb|b )| возвращает некоторое значение (\verb|action1| или \verb|action2|), которое сохраняется в переменной \verb|val1|, значение нетерминала сохраняется в переменной \verb|val2|, и далее обе эти переменные передаются в качестве аргументов в пользовательскую функцию \verb|someFunc|.

Возможность таким способом задавать атрибуты вызывает сложности при интерпретации дерева вывода. Связаны они с тем, что для вычисления атрибутов становится недостаточно информации только о дереве вывода входного выражения. 

Рассмотрим эту проблему более подробно. Допустим, в грамматике есть правило:

\begin{verbatim}
  someRule : val1 = (a {action1})* val2 = c  {someFunc val1 val2};
\end{verbatim}

Узел дерева вывода, соответствующий правилу, приведённому выше, может выглядеть следующим образом:

\begin{centering}
  \begin{dot2tex}[dot,autosize]

  digraph string_of_child
  {
          a1[label = "a"];
          a2[label = "a"];
          a3[label = "a"];
          c[label = "c"];
          S[label = "someRule"]
            
          S -> a1;
          S -> a2;
          S -> a3;
          S -> c;                            
  }
  \end{dot2tex}

\end{centering} 

Для вычисления атрибутов в этом узле необходимо знать,  что первые три сына были порождены из замыкания, их необходимо объединить в список и уже его передать в функцию \verb|someFun| в качестве первого параметра.

В общем случае можно  рассматривать непосредственных сыновей узла как строку, принадлежащую языку, задаваемому регулярным выражением в правой части правила.  Тогда можно сказать, что для вычисления атрибутов нам необходимо дерево разбора этой строки.

Для нашего примера:

\begin{centering}
  \begin{dot2tex}[dot,autosize]

  digraph string_of_child
  {
            S[label = "someRule"]

            c[label = "c"]; 
            a1[label = "a"];
            a2[label = "a"];
            a3[label = "a"];           
              
            S -> c;                            
            S -> a1;
            S -> a2;
            S -> a3;

          subgraph cluster_STR
          {                                                
                  bgcolor = grey;
                  str[label = "1",texlbl = "$Str:$",shape = plaintext]
                  c
                  a1;
                  a2;
                  a3;
                  
          };
  }
  \end{dot2tex}
%\captionof{figure}{ntcn}
%	\label{fig:rr}

\end{centering}
 
Где $Str$ -- строка, принадлежащая языку, задаваемому регулярным выражением в правой части правила. Дерево разбора этой строки будет выглядеть так:

\begin{centering}
  \begin{dot2tex}[dot,autosize]

  digraph string_diriv_tree
  {

            Seq[label = "Seq"]
            Cls[label = "Cls"]
            a1[label = "a"];
            a2[label = "a"];
            a3[label = "a"];
            c[label = "c"];                        
                                   
            Seq -> Cls;            
            Seq -> c; 
            Cls -> a1;
            Cls -> a2; 
            Cls -> a3;                           

  }
  \end{dot2tex}
  %\captionof{figure}{ntcn}
  %	\label{fig:rr}

\end{centering}

Таким образом, для того, чтобы вычислять атрибуты, нам необходимо во время интерпретации дерева вывода входного выражения построить дерево разбора строки из сыновей узла, для которого непосредственно производятся вычисления. Для этого необходимо во время анализа поучить и сохранить информацию о выводе этой строки. 

Удобным механизмом для решения этой проблемы оказался конечный автомат с помеченными переходами (далее будем для простоты будем называть их LFA -- Labelled Finite Automaton). Общая схема решения такова: 
\begin{enumerate}
	\item строится недетерминированный конечный автомат с помеченными переходами (LNFA), в котором в качестве меток сохраняется информация о начале и завершении конструкций регулярного выражения, по которому строится этот LNFA;
  \item по LNFA строится детерминированный конечный автомат с помеченными переходами (LDFA);
  \item в процессе анализа собираются и сохраняются метки с совершённых переходов и на основе этой информации строится дерево разбора.
\end{enumerate}

Подробнее все эти шаги будут описаны далее.



\subsubsection{Построение LNFA по регулярному выражению}

Определим LNFA как шестёрку $(Q$, $\Sigma$, $L$, $T$, $q_0$, $F)$, состоящая из:
\begin{itemize}
	\item конечного множества состояний $Q$ 
	\item конечного множества входных символов $\Sigma$ 
	\item конечного множества меток $L$ 
	\item функции перехода $T: \; Q \times (\Sigma \cup{ \varepsilon })\rightarrow 2^{Q \times L}$
	\item начального состояния $q_0 \in Q$
	\item конечного множества финальных состояний $F \subseteq Q$ 
\end{itemize}

Таким образом переходы автомата снабжаются метками. При изображении автомата в виде графа переходам соответствуют рёбра. Будем записывать метки через знак "/" \; после символа, принимаемого автоматом при данном переходе. 

Пример LNFA:

\begin{dot2tex}[dot]
digraph G
{
        rankdir = LR
        F [shape = doublecircle]
        S -> F [  label="a/l"
                , texlbl = "$a/someLbl$" ]
}
\end{dot2tex}


Где $a$ -- принимаемый символ, $someLbl$ -- метка.

Чтобы решить задачу вычисления атрибутов необходимо знать, когда началось и когда закончилось распознавание той или иной конструкции регулярного выражения. Для этого определим  метки специального типа:
\begin{itemize}
	\item для обозначения начала и конца конструкции
		\begin{itemize}
			\item лист: $LeafS,$ $LeafE$;
			\item последовательность: $SeqS,$ $SeqE$;
			\item замыкание: $ClsS,$ $ClsE$;
			\item альтернатива: $Alt1S,$ $Alt1E,$ $Alt2S,$ $Alt2E$ - пара меток для каждой ветви;
		\end{itemize}
			\item $\omega$ -- "`пустая"' метка;
\end{itemize}

Метка для конкретного ребра будет состоять из типа метки и уникального идентификатора, который совпадает у меток начала и конца одной и той же конструкции. Таким образом, множество меток $L$ можно определить так:

\begin{eqnarray}
     \label{def:L}
	   &L = \left\{ \right. t*k\; | \; t \in \left\{ \right.  LeafS,\; LeafE,\; SeqS,\; SeqE,\; ClsS,\; ClsE, & \nonumber \\
	   & \qquad Alt1S,\; Alt1E,\; Alt2S,\; Alt2E,\; \omega \; \left. \right\},\; k \in N \left.\right\} &
\end{eqnarray}

Для построения LNFA по регулярному выражению необходимо модернизировать алгоритм Томпсона. Его необходимо дополнить механизмом расстановки меток.

Будем рассматривать алгоритм, который работает с деревом вывода регулярного выражения, которое мы можем получить из YARD-а. Это дерево может содержать следующие конструкции:
\begin{itemize}
  \item Leaf(a) -- лист дерева. Соответствует символу в регулярном выражении.
  \item Seq(lst) -- последовательность. lst -- список элементов последовательности.
  \item Alt(L,R) -- альтернатива.
  \item Cls(T) -- замыкание.
\end{itemize}

Определим ряд функций:
\begin{itemize}
  \item $\mathop{buildL\!N\!F\!A:} \; 'Tree \rightarrow {'L\!N\!F\!A}$ -- строит LNFA по дереву вывода регулярного выражения.

  \item $map: \; ('T \rightarrow {'U}) \rightarrow {'T}  \; list \rightarrow {'U} \; list$ -- применяет функцию, переданную в качестве первого аргумента, к каждому элементу списка, перданного вторым аргументом.

  \item $concat: \; 'L\!N\!F\!A \; list \rightarrow {'L\!N\!F\!A}$ -- конкатенирует автоматы из списка, добавляя $\varepsilon/\omega$-переходы.  
\end{itemize}

Так же предположим, что у нас есть функция для генерации уникального индекса $k$ для нумерации меток.

Модернизированный алгоритм будет выглядеть следующим образом:
  \begin{itemize}
    \item
      Лист : Leaf(a) \
      \begin{flushleft}
        \begin{dot2tex}[dot]

digraph createTNFALeaf
{
    rankdir = LR;    
    i_0[ texlbl = "$i$"];
    i_1[ texlbl = "$i+1$"];
    i_2[ texlbl = "$i+2$"];
    i_3[ texlbl = "$i+3$"];

    i_0 -> i_1[label="a/e", texlbl = "$\epsilon/(LeafS,k)$"];
    i_1 -> i_2[label="a/e", texlbl = "$a/\omega$"];
    i_2 -> i_3[label="a/e", texlbl = "$\epsilon/(LeafE,k)$"];

}

\end{dot2tex}

      \end{flushleft}
    \item 
      Последовательность : Seq(lst) \
      \begin{flushleft}
        \begin{dot2tex}[dot]

digraph createTNFASeq
{
  rankdir = LR;

  map[ texlbl = "$concat$ ($map$ $buildL\!N\!F\!A$ lst)"
     , shape = box
     , label = "concat (map buildTNFA lst)"];
  

  i[ texlbl = "$i$"];

  j[ texlbl = "$i+1$"]
  
  i -> map[ texlbl = "$\epsilon/(SeqS,k)$"
          , label = "123"];
  map -> j[ texlbl = "$\epsilon/(SeqE,k)$"
          , label = "123"];
}

\end{dot2tex}

      \end{flushleft}
    \item 
      Альтернатива : Alt(L,R) \
      \begin{flushleft}
        \begin{dot2tex}[dot]

digraph createTNFAAlt
{
  rankdir = LR;

  rankdir = LR;

  doA[ texlbl = "($buildL\!N\!F\!A$ L)"
        , shape = box
        , label = "(buildTNFA L)"];

  doB[ texlbl = "($buildL\!N\!F\!A$ R)"
        , shape = box
        , label = "(buildLNFA R)"];

  i_2[ texlbl = "$i$"
   ];

  j_2[ texlbl = "$i+1$"
   ]
  
  i_2 -> doA[ texlbl = "$\epsilon/(Alt1S,k)$"
            , label = "123"];
  i_2 -> doB[ texlbl = "$\epsilon/(Alt2S,k)$"
            , label = "123"];

  doA -> j_2[ texlbl = "$\epsilon/(Alt1E,k)$"
            , label = "12345"];
  doB -> j_2[ texlbl = "$\epsilon/(Alt2E,k)$"
            , label = "12345"];
}

\end{dot2tex}

      \end{flushleft}
    \item 
      Замыкание : Cls(T) \
      \begin{flushleft}
        \begin{dot2tex}[dot]

digraph createTNFACls
{
rankdir = LR;

  subgraph Cls 
  {
   
    rankdir = LR;

    i[ texlbl = "$i$"];

    subgraph L{
    e_1[ texlbl = "$i+1$"];

    doA_2[ texlbl = "($buildL\!N\!F\!A$ T)"
         , shape = box
         , label = "a"];

    e_2[ texlbl = "$i+2$"];
    }
    j_3[ texlbl = "$i+3$"];

    i -> e_1[ texlbl = "$\varepsilon/(ClsS,k)$"
            , label = "1"];
    
    e_1 -> doA_2[ texlbl = "$\varepsilon/ \omega$"
                , label = " "];
    doA_2 -> e_2[ texlbl = "$\varepsilon/ \omega$"
                , label = " "];
    e_2 -> e_1[ texlbl = "$\varepsilon/ \omega$"
              , label = " "];
    e_1 -> e_2[ texlbl = "$\varepsilon/ \omega$"
              , label = " "];
                    
    e_2 -> j_3[ texlbl = "$\varepsilon/(ClsE,k)$"
              , label = "1"];
     
  }
 
}
\end{dot2tex}

      \end{flushleft}
  \end{itemize}

Таким образом, теперь мы можем по регулярному выражению построить LNFA с метками, соответствующими началу и концу каждой конструкции регулярного выражения.


\subsubsection{Построение LDFA по LNFA}

Для дальнейшего использования необходимо преобразовать недетерминированный автомат в детерминированный автомат. Для этого можно использовать стандартный алгоритм построения DFA по NFA~\cite{DrgBook}, расширенный для работы с метками.

Определим детерминированный конечный автомат с метками (LDFA), как как шестёрку $(Q$, $\Sigma$, $L'$, $T$, $q_0$, $F)$, состоящая из:
\begin{itemize}
	\item конечного множества состояний $Q$ 
	\item конечного множества входных символов $\Sigma$ 
	\item конечного множества меток $L$ 
	\item функции перехода $T: \; Q \times \Sigma \rightarrow Q \times L'$
	\item начального состояния $q_0 \in Q$
	\item конечного множества финальных состояний $F \subseteq Q$ 
\end{itemize}

В нашем случае $L' = 2^{(2^L)}$, где $L$ -- множество меток LNFA $\eqref{def:L}$ и подмножества $L$ являются упорядоченными.

Процесс построения детерминированного автомата с помеченными переходами можно разбить на 2 этапа:
\begin{enumerate}
	\item построение детерминированного автомата с помощью стандартного алгоритма;
	\item вычисление и расстановка новых меток.
\end{enumerate}

Рассмотрим второй этап более подробно. Сперва необходимо разбить состояния DFA, построенного на первом шаге, на два множества: $F$ -- множество конечных состояний и $I = Q/F$ -- остальные (не конечные) состояния. 

Для дальнейшей работы нам понадобится функция $calculateN\!ewLabel:\; {'\!state} \rightarrow '\!l$, где $'\!l \in L'$. Она будет по состоянию автомата вычислять метку перехода из этого состояния. Определим эту функцию следующим образом:
\\
$calculateN\!ewLabel \; state \; = $\\
$\phantom \qquad let \; e = \text{множество всех } \varepsilon\text{-цепочек, заканчивающихся в state} $ \\
$\phantom \qquad let \; l = map \ (fun \ eLine \rightarrow \text{ пройти по } eLine \text{ из начала в конец}$ \\
$\phantom \qquad \qquad \quad \text{ и собрать по порядку все метки}) \; e \\$
$\phantom \qquad l$\\
Основная функция для вычисления и расстановки новых меток: \\
$setN\!ewLabel = $\\
$ \phantom \qquad \text{Для каждого состояния } f \in F \text{ добавить новое "`висячее"' \ ребро с началом в } f$ \\
$ \phantom \qquad \text{и установить ему метку равную } (calculateN\!ewLabel \ f)$\\
$ \phantom \qquad \text{Для каждого состояния } i \in I \text{, для каждого ребра, выходящего из } i$ \\
$ \phantom \qquad \text{установить метку равную } (calculateN\!ewLabel \ i)$

Рассмотрим работу данного алгоритма на примере. Пусть дано правило грамматики: $S \rightarrow a|b$. Необходимо построить ДКА с помеченными переходами для этого правила. По регулярному выражению в правой части ,применив алгоритм, описанный выше, построим НКА. В результате мы получим следующий автомат:
 
\begin{centering}

  \begin{dot2tex}[dot]
  digraph G
  {
          1 [label = "1"]
          2 [label = "2"]
          3 [label = "3"]
          4 [label = "4"]
          5 [label = "9"]
          6 [label = "10", shape = doublecircle]
          15 [label = "5"]
          16 [label = "6"]
          17 [label = "7"]
          18 [label = "8"]


          5 -> 15[label = "1234567", texlbl = "$\varepsilon/(Alt1S,1)$"]
          5 -> 17[label = "1234567", texlbl = "$\varepsilon/(Alt2S,1)$"]
          15 -> 1[label = "1234567", texlbl = "$\varepsilon/(LeafS,1)$"]
          1 -> 2 [label = "123", texlbl = "$a/\omega$"]
          2 -> 16[label = "1234567", texlbl = "$\varepsilon/(LeafE,1)$"]
          17 -> 3[label = "1234567", texlbl = "$\varepsilon/(LeafS,2)$"]
          3 -> 4 [label = "123", texlbl = "$b/\omega$"]
          4 -> 18[label = "1234567", texlbl = "$\varepsilon/(LeafE,2)$"]
          16 -> 6[label = "1234567", texlbl = "$\varepsilon/(Alt1E,1)$"]
          18 -> 6[label = "1234567", texlbl = "$\varepsilon/(Alt2E,1)$"]
  }
  \end{dot2tex}

\end{centering}

Далее, по этому автомату строится ДКА. Для построения применяется стандартный алгоритм Томпсона. Метки на рёбрах можно опустить, они будут вычислены на следующем шаге. Результирующий автомат будет выглядеть следующим образом:

\begin{centering}

  \begin{dot2tex}[dot]
  digraph G 
  {
    1 
    2 [shape = doublecircle]
    3 [shape = doublecircle]
    1 -> 2[ label = "a"]
    1 -> 3[ label = "b"]
  }
  \end{dot2tex}

\end{centering}

После этого можно вычислить новые метки, способом описанным выше. Автомат изменится -- в нём появятся две "`висячие"' \ дуги -- для каждого из финальных состояний. Выглядеть результирующий автомат будет следующим образом: 

\begin{centering}

  \begin{dot2tex}[dot]
  digraph G 
  {
    1 
    2 [shape = doublecircle]
    3 [shape = doublecircle]
    4 [shape = none, label = ""]
    5 [shape = none, label = ""]
    1 -> 2[ label = "01234567901234", texlbl = "$a/[[(Alt1S,1); \ (LeafS,1)]]$"]
    1 -> 3[ label = "01234567901234", texlbl = "$b/[[(Alt2S,1); \ (LeafS,2)]]$"]
    2 -> 4[ label = "01234567901234", texlbl = "$\varepsilon/[[(LeafE,1); \ (Alt1E,1)]]$"]
    3 -> 5[ label = "01234567901234", texlbl = "$\varepsilon/[[(LeafE,2); \ (Alt2E,1)]]$"]
  }
  \end{dot2tex}

\end{centering}

Видно, что метками в полученном детерминированном автомате являются множества списков меток недетерминированного автомата. 

Теперь рассмотрим алгоритм, с помощью которого по построенному LDFA можно получить нужные нам данные о выводе входной строки.
\begin{enumerate}
   	\item $let \ L\!D\!F\!A = $ LDFA, построенный по регулярному выражению, как описано выше.
   	\item $let \ labelsList = []$ (*список для хранения меток, полученных в процессе работы автомата*)
    \item $let \ trace = []$ (*трасса работы автомата*)    
   	\item Запускаем $L\!D\!F\!A$ над входной строкой. При каждом переходе добавляем метку $l_i$ в начало $labelsList$.
   	\item (*После завершения работы $L\!D\!F\!A$ в $labelsList$ хранится инвертированная последовательность меток.*)
   	\item $if \ \text{строка принята}$
   	\item $then$ 
   	\item $\phantom \qquad$Просматриваем $labelsList$ из начала в конец.
   	\item $\phantom \qquad let \ l_i  = $ текущий элемент $labelsList$
   	\item $\phantom \qquad let \ l_{i+1}  = $ следующий элемент $labelsList$
    \item $\phantom \qquad$Добавляем в $trace \ (k \ | \ k \ \in \ l_i: \exists p \in l_{i+1}: \\ \text{первый} \text{ элемент из } p \text{  является закрывающим для последнего элемента из } k)$
   	\item $else$ 
   	\item $\phantom \qquad Error$
   	\item $reverse \ trace$
\end{enumerate}

Таким образом, мы построили детерминированный конечный автомат с помеченными переходами, с помощью которого можно получить необходимую для дальнейшей работы информацию о выводе входной строки.


\subsubsection{Генерация кода для семантических действий пользователя}

При задании грамматики пользователь может, пользуясь атрибутами, указывать семантические действия. Необходимо на основе этих атрибутов построить код, который будет производить необходимые вычисления.

Для вычисления атрибутов был выбран подход интерпретации дерева (леса) вывода. Основная идея этого алгоритма -- обход заранее построенного дерева вывода и вычисление необходимых функций в его узлах.

В нашем случае дерево будет обходиться снизу вверх, так на данном этапе было решено поддержать только s-атрибутные грамматики. 

При интерпретации дерева вывода оно обходится снизу-вверх и в каждом узле вычисляется некоторая функция, основанная на атрибутах из пользовательской грамматики. Каждый внутренний узел в дереве вывода выражения в данной грамматике соответствует правилу из этой грамматики. Поэтому для интерпретации дерева достаточно построить функцию, соответствующую правилу грамматики и основанную на атрибутах, заданных в этом правиле. При обходе дерева в каждом его узле будет вычисляться соответствующая ему функция.

Функция всегда принимает один аргумент -- дерево разбора строки из сыновей узла, в котором она вычисляется, и является интерпретатором этого дерева. Результат вычислений помещается в узел, для которого функция вычислялась.

Параллельно с генерацией кода строится таблица соответствий между функцией и правилом, для которого она построена. Эта таблица будет применяться при интерпретации дерева вывода для поиска функции, соответствующей текущему узлу дерева. 


\subsubsection{Интерпретация дерева вывода}

Чтобы вычислить семантические действия, заданные пользователем, необходимо выполнить интерпретацию дерева вывода. Для этого нужно обойти дерево снизу вверх и в каждом внутреннем узле вычислить соответствующую функцию, которая ищется с помощью таблицы соответствий между правилом и функцией.

Общая схема интерпретации дерева выглядит следующим образом:
\begin{itemize}
  \item построенное дерево разбора обходится снизу вверх;
  \item для очередного внутреннего узла, на основе трассы, хранимой в нём, строится дерево разбора строки из сыновей;
  \item с помощью таблицы, построенной на этапе генерации, ищется функция, соответствующая данному узлу;
  \item найденная функция применяется к построенному дереву разбора;
  \item результат сохраняется в текущем узле;
\end{itemize} 

В процессе анализа в узлах дерева вывода накапливается информация, необходимая для построения дерева разбора строки из его сыновей в виде трасс вычислений соответствующих автоматов.

Полученная трасса является, по сути своей, правильной скобочной структурой, которую надо наложить на строку из сыновей. Сделать это не сложно, так как каждый символ строки окружён скобками. После этого становится возможным построить дерево разбора строки сыновей. Для этого скобочная пара сворачивается в узел дерева, а все элементы, лежащие внутри скобок, превращаются в сыновей этого узла.

Дерево разбора будет содержать четыре типа узлов:\\
$type \ aTree< \ '\!value> \ = \\
\phantom \qquad |\ ASeq \ of \ List<\!aTree\!>\\
\phantom \qquad |\ AAlt \ of \ Option<\!aTree\!> * Option<\!aTree\!>\\
\phantom \qquad |\ ACls \ of \ List<\!aTree\!>\\
\phantom \qquad |\ ALeaf \ of \ 'value\\
$

Алгоритм построения дерева выглядит следующим образом:
\\
$buildTree \ line = \\
\phantom \qquad let \ group = \ \text{первая скобочная пара из line}\\ 
\phantom \qquad let \ end = \ \text{line без group} \\
\phantom \qquad let \ tree = match \ \text{тип внешних скобок group} \ with\\
\phantom \qquad \qquad |\ Seq \rightarrow    ASeq( buildTree \ \text{значение внутри скобок group})\\
\phantom \qquad \qquad |\ Alt1 \rightarrow   AAlt( (buildTree \ \text{значение внутри скобок group}),N\!one)\\
\phantom \qquad \qquad |\ Alt2 \rightarrow   AAlt( N\!one, (buildrTree \ \text{значение внутри скобок group}))\\
\phantom \qquad \qquad |\ Cls \rightarrow    ACls( buildTree \ \text{значение внутри скобок group})\\
\phantom \qquad \qquad |\ Leaf \rightarrow ALeaf(\text{ значение внутри скобок group)}   \\
\phantom \qquad if \ end \ \text{пусто} \\
\phantom \qquad then \ [tree] \\
\phantom \qquad else \ tree::(buildTree \ end)\\
$

Предложенный выше алгоритм, позволяет вычислять пользовательские атрибуты, обеспечивая при этом непосредственную поддержку EBNF-грамматик. 

\subsection{Архитектура}
%\section{Архитектура}

В ходе работы был реализован прототип генератора синтаксических анализаторов, основанный на описанном выше алгоритме. В процессе изучения алгоритма было выяснено, что его удобно реализовывать на функциональном языке программирования, поэтому  прототип реализован на функциональном языке программирования для платформы .NET -- F\#. Общая схема реализованного инструмента приведена ниже.


\begin{center}
  \includegraphics[height = 14.5cm]{general_tool_structure.pdf}
	\captionof{figure}{Общая структура инструмента}
	\label{fig:general_tool_structure}
\end{center}

Реализованный инструмент состоит из трёх основных модулей:
\begin{itemize}
	\item {\bfseries Парсер входной грамматики} (фронтенд), который основан на инструменте YARD, из которого используется входной язык (соответственно лексический и синтаксический анализаторы этого языка), внутреннее представление языка описания грамматики и набор преобразований (используется в генераторе). 
	
	\item {\bfseries Генератор}, который по дереву грамматики строит набор таблиц и генерирует функции для вычисления пользовательских атрибутов. Основные этапы:
		\begin{itemize}
			\item {\bfseries Преобразования грамматики}, которые необходимы, так как на уровне входного языка YARD поддерживает конструкции, которые на текущий момент наш инструмент не поддерживает(например макроправила).
				\item {\bfseries Генерация таблиц и кода}
		\end{itemize}
		Результатами работы генератора являются:
		\begin{itemize}
			\item {\bfseries Набор таблиц}, который содержит данные, необходимые для синтаксического анализа и вычисления атрибутов и состоит из:
			\begin{itemize}
				\item {\bfseries GOTO} -- таблица переходов синтаксического анализатора;
		  	\item {\bfseries Items} -- информация о состояниях  синтаксического анализатора;
		  	\item {\bfseries Action Map} -- отображение из правил в функции,  сгенерированные по атрибутам этого правила;
			\end{itemize}
			
		  \item {\bfseries Функции вычисления атрибутов } -- файл на F\#, содержащий функции для вычисления пользовательских атрибутов.
		\end{itemize}
		
		
	\item {\bfseries Ядро}, которое реализует синтаксический разбор и вычисление атрибутов и содержит:
		\begin{itemize}
			\item {\bfseries Интерпретатор таблиц}, который строит лес вывода входного выражения на основе сгенерированных таблиц. Основан на рекурсивно-восходящем алгоритме. Возвращает лес -- список деревьев вывода.
			\item {\bfseries Вычислитель атрибутов}, который обходит лес, полученный от интерпретатора таблиц, и применяет функции, найденные с помощью Action Map в сгенерированном файле, к узлам дерева.
		\end{itemize}
		
\end{itemize}

На вход поступает файл с грамматикой, заданной пользователем и поток лексем. На выходе -- результат вычислений, заданных пользователем в атрибутах грамматики.


\clearpage
\section{Эксперименты}

Был проведён ряд экспериментов, которые позволили оценить некоторые параметры инструмента.

Во всех приведённых ниже тестах грамматики описываются на языке YARD. Так же предположим, что у нас есть сторонний лексер со следующим набором лексем:
\begin{itemize}
  \item PLUS = '+'
  \item MINUS = '-'
  \item DIV = '/'
  \item MULT = '*'
  \item LEFT = '('
  \item RIGHT = ')'
  \item NUMBER = (0..9)+
\end{itemize}

По этому будем предполагать, что на вход инструменту поступает поток лексем.


\subsection{Работа с однозначными грамматиками} 

Необходимо показать, что по однозначной грамматике строится инструмент имеющий линейную временную сложность.
	<Пример грамматики> <Описание эксперимента>


\subsection{ Возможность работы с неоднозначными грамматиками} 

Для этого необходимо проверить, что при неоднозначной грамматике инструмент возвращает все возможные варианты вывода входной строки. В качестве примера была взята следующая грамматика:

\begin{verbatim}
+s : e;
e : e (PLUS | MINUS | MULT | DIV  ) e 
  | LEFT e RIGHT 
  | NUMBER ;
\end{verbatim}

Эта грамматика описывает арифметические выражения без приоритетов. Очевидно, что данная грамматика содержит неоднозначности. 

Рассмотрим несколько примеров входных цепочек:
\begin{itemize}

  \item \verb|[NUMBER; PLUS; NUMBER]|. Существует единственное дерево вывода для данной цепочки:
    \begin{centering}
      \begin{dot2tex}[dot]
       digraph g
       {
          S [label = "S"]
          e1 [label = "e"]
          e2 [label = "e"]
          plus [label = "PLUS"]
          e3 [label = "e"]
          num1 [label = "NUMBER"]
          num2 [label = "NUMBER"]
          S -> e1
          e1 -> e2
          e1 -> plus
          e1 -> e3
          e2 -> num1
          e3 -> num2
       }
      \end{dot2tex}
    \end{centering}

Инструмент возвращает единственное дерево:
\begin{verbatim}
<NODE name="S">
        <NODE name="e">
            <NODE name="e">
                <LEAF name="NUMBER"/>
            </NODE>
            <LEAF name="PLUS"/>
            <NODE name="e">
                <LEAF name="NUMBER"/>
            </NODE>
        </NODE>
</NODE>
\end{verbatim}

\item \verb|[NUMBER; PLUS; NUMBER; PLUS; NUMBER]|. Для данной цепочки существует два дерева вывода и инструмент возвращает оба:
\begin{verbatim}
<NODE name="s">
    <NODE name="e">
        <NODE name="e">
            <NODE name="e">
                <LEAF name="NUMBER"/>
            </NODE>
            <LEAF name="PLUS"/>
            <NODE name="e">
                <LEAF name="NUMBER"/>
            </NODE>
        </NODE>
        <LEAF name="PLUS"/>
        <NODE name="e">
            <LEAF name="NUMBER"/>
        </NODE>
    </NODE>
</NODE>

<NODE name="s">
    <NODE name="e">
        <NODE name="e">
            <LEAF name="NUMBER"/>
        </NODE>
        <LEAF name="PLUS"/>
        <NODE name="e">
            <NODE name="e">
                <LEAF name="NUMBER"/>
            </NODE>
            <LEAF name="PLUS"/>
            <NODE name="e">
                <LEAF name="NUMBER"/>
            </NODE>
        </NODE>
    </NODE>
</NODE>
\end{verbatim}
	
\end{itemize}
	


\subsection{Возможность работы с EBNF-грамматиками} 

Основные конструкции регулярных выражений, для которых необходимо провести проверку - последовательность, альтернатива, замыкание. Важно обратить внимание на соответствие получаемого дерева вывода ожидаемому результату. Для этого нужно проверить соответствие дерева вывода входной грамматике. В нём не должно быть новых терминалов и нетерминалов.
	


\subsection{Поддержка s-атрибутных грамматик}

 Необходимо показать корректность вычисления атрибутов в случае неоднозначной грамматики. Для этого необходимо, чтобы были получены все возможные результаты вычислений и чтобы операции с побочными эффектами работали корректно (например, проверить, что при наличии в атрибутах действия печати на экран, на экран  не выводится лишней информации). Так же необходимо показать возможность вычисления атрибутов в случае расширенной контекстно-свободной  грамматики. 
\\
<Описание эксперимента>
\\

Так же, в ходе этого эксперимента было выявлено, что предложенное решение, когда явным образом строится дерево вывода и для каждого правила строится своя семантическая функция, оказывается удобным. С одной стороны, это позволяет упростить отладку, потому, что всегда можно проверить правильность построения дерева и в отладчике просто проконтролировать вычисление в конкретном узле (мы знаем при свёртке какого правила появился этот узел и знаем какая функция должна вычисляться). С другой -- прямой доступ к лесу вывода позволяет совершать с ним дополнительные операции.  Дополнительную фильтрацию или, например печать, что оказалось полезным при получении результатов экспериментов (печать XML-представления деревьев). 

\clearpage
\section{Заключение}

В ходе выполнения данной дипломной работы были получены следующие результаты:
\begin{itemize}
	%\item выяснено, что предпочтительным алгоритмом анализа является GLR алгоритм;
	
	%\item изучен рекурсивно-восходящий алгоритм анализа;
	
	\item реализован прототип генератора синтаксических анализаторов со следующими свойствами:
		\begin{itemize}
		  \item принимает на вход s-атрибутную грамматику YARD-а и строит по ней транслятор
		  
			\item по однозначной LR-грамматике строится анализатор с линейной сложностью;
			
			\item по неоднозначной грамматике строится анализатор, возвращающий все возможные деревья вывода для данной входной цепочки;
			
			\item реализует поддержку EBNF-грамматик без их преобразования;
			
			\item реализует вычисление s-атрибутов; 
		
		\end{itemize}
		
\end{itemize}

Кроме того, в ходе экспериментов было выяснено, что предложенный подход, при котором явно строится лес вывода, и функции вычисления атрибутов соответствуют правилам грамматики, сильно упрощает отладку целевого инструмента.

\subsection{Дальнейшее развитие}

   Количество конструкций регулярных выражений, применяемых на практике при описании грамматики, достаточно велико, однако в инструменте поддержаны далеко не все. Ещё одним направлением развития является поддержка наследуемых атрибутов, применение которых сильно упрощает трансляцию с учётом контекста~\cite{Diploma}. Так же необходимо реализовать поддержку предикатов (резольверов~\cite{MartinenkoSUT}). Стоит отметить, что все эти возможности реализованы на уровне входного языка инструмента YARD.

Кроме того необходимо дополнить инструмент такими ставшими привычными средствами, как автоматическое восстановление от ошибок и средства диагностики ошибок.

\clearpage
\begin{thebibliography}{50}

        \bibitem {Reeng} ������������������ ������������ �������� / ��� ���. ����. �.�. �������� � �.�. ��������. - ���.: ������������ �.-�������������� ������������, 2000. 332~�.

        \bibitem {DrgBook} \emph {��� �., ���� �., ������ ��.} �����������: ��������, ����������, �����������.  �:. ������������ ��� <�������>2003. 768~�.

        \bibitem {Diploma} \emph{��������� �.�.} ��������� �������������� ������������  ��� ������� ����� ������������������� ������������� ��������. 2007. 37~c.        

        \bibitem {CCReview} \emph{��������� �.�., ������ �.�.} ����� ����������� ������� ������������� �������� �������������� ������������ // ��������� ����������������. - ���.: ���-�� �.-������. ��-��, 2006. 286-316~�.


        \bibitem {Practical Guide} \emph{Dick Grune, Ceriel Jacobs} PARSING TECHNIQUES A Practical Guide

        \bibitem {RECURSIVE-ASCENT PARSING} \emph {Larry Morell, David Middleton} RECURSIVE-ASCENT PARSING. Arkansas Tech    
                 University Russellville, Arkansas. 

        \bibitem {RecursiveAscentParsing} \emph {Lex Augusteijn} Recursive Ascent Parsing (Re: Parsing techniques). lex@prl.philips.nl (Lex Augusteijn) Mon, 10 May 1993 07:03:39 GMT 


        \bibitem {CurrentParsTechn} \emph{Mark G.J. van den Brand, Alex Sellink, Chris Verhoef} 
                Current Parsing Techniques in Software Renovation Considered Harmful.// IWPC '98: Proceedings of the 6th International Workshop on Program Comprehension. - IEEE Computer Society, Washington,1998.
        
        \bibitem {ISOEBNF} \emph ISO/IEC 14977 : 1996(E)

        \bibitem {Non-det-rec-asc} \emph {Rene Leermakers} Non-deterministic Recursive Ascent Parsing. Philips Research Laboratories,
                 P.O. Box 80.000, 5600 JA Eindhoven, The Netherlands. 

        


        \bibitem {Jade} \emph {Ronald Veldena} Jade, a recursive ascent LALR(1) parser generator. September 8,1998


        \bibitem {Bison}    http://www.gnu.org/software/bison (���� ������������� Bison)

        \bibitem {ASF+SDF}  http://www.meta-environment.org (���� ������������� ASF+SDF)

        \bibitem {.NET}     http://www.microsoft.com/NET/ (���� ��������� .NET)  

        \bibitem {FS}       http://www.research.microsoft.com/fsharp (������������ � ������������ �� ����� F\#)             

        \bibitem {Elkhound} http://www.scottmcpeak/elkhound/ (���� ������������� Elkhound )

\end{thebibliography}


\end{document}
