\section{Введение}	

Задачи автоматизированного  реинжиниринга  программ~\cite{Reeng} выдвигают особые требования~\cite{CurrentParsTechn} к генераторам синтаксических анализаторов. Во многом это связано с тем, что теория синтаксически управляемой трансляции развивалась одновременно языками, сейчас называемых устаревшими (legacy languages). Тогда еще не были получены основные теоретические результаты, положенные в основу наиболее распространенных современных генераторов синтаксических анализаторов. Поэтому устаревшие языки имеют особенности, затрудняющие синтаксический анализ даже с использованием современных инструментов (например ASF+SDF~\cite{ASF+SDF}, Elkhound~\cite{Elkhound}), которые значительно упрощают создание анализаторов.

Для устаревшего языка сложно (а зачастую и невозможно) задать однозначную контекстно-свободную грамматику. Необходимо существенно преобразовать его спецификацию, которая приводится в документации, чтобы получить такую грамматику. Но после этого она серьёзно усложняется и на её сопровождение требуется больше ресурсов~\cite{CurrentParsTechn}. Поэтому устаревший язык обычно задается с помощью неоднозначной контекстно-свободной грамматики.

При решении задач реинжиниринга часто требуются преобразования уже существующей грамматики. С одной стороны, при разработке  грамматики могут быть допущены неточности, с другой, документация устаревших языков и, в особенности, их диалектов может содержать ошибки, быть неполной или вообще отсутствовать. В результате, в целевом инструменте появляются ошибки, многие из которых возможно выявить только на этапе его тестирования. Для их исправления необходимо корректировать исходную грамматику. Кроме этого, при поддержке нового диалекта, как правило, проще изменить уже разработанную грамматику другого диалекта, а иногда может оказаться удобным описать несколько диалектов в рамках одной грамматики, просто "дополнив" исходную. При этом,зачастую, изменение одного правила приводит к появлению десятков конфликтов в грамматике~\cite{CurrentParsTechn}, которые необходимо разрешать "`вручную"', что требует большого количества времени. 

Для решения этих задач предлагается использовать неоднозначные контекстно-свободные грамматики и соответствующие инструменты построения анализаторов~\cite{CurrentParsTechn}. Основная особенность этих инструментов -- алгоритм, который способен работать с неоднозначными грамматиками(GLR-алгоритм). Анализатор, построенный по неоднозначной  грамматике с помощью данного алгоритма, в общем случае, в результате разбора строит не единственное дерево, а несколько деревьев -- лес, содержащий все возможные варианты вывода. Дальнейшая работа с полученным лесом организуется исходя из требований и  особенностей решаемой задачи. Это избавляет он необходимости "`ручного"' устранения конфликтов, что существенно сокращает время и упрощает разработку грамматики. Важным плюсом является ещё и то, что код становится более компактным и сопровождаемым.
      
%который можно сократить, используя специальные фильтры, а можно ,при задании в одной спецификации нескольких диалектов, вернуть весь лес для дальнейшего выбора нужного дерева/диалекта.   

%Работу GLR-алгоритма  можно рассматривать как параллельное исполнение набора LR-анализаторов. При этом данный набор дополняется процедурой управления стеками, оптимизирующей представление стеков путем их «склеивания» и «расклеивания», что позволяет хранить и строить параллельные выводы в рамках одного LR-анализатора, лишь в моменты их различия добавляя параллельный анализатор.

%Оказалось, что весьма наглядно такой алгоритм может быть представлен в виде набора взаимно-рекурсивных функций ~\cite{Non-det-rec-asc},~\cite{RECURSIVE-ASCENT PARSING},~\cite{RecursiveAscentParsing}. При этом расклеивание стека получается естественным образом как ветвление в одной из функций, а обратное склеивание может быть реализовано как кэширование результата функции.

Стоит отметить, что по производительности такой анализатор, являясь некоторой "надстройкой" \ над LR-анализатором, незначительно ему уступает. На сегодняшний день в соотношении производительность/класс разбираемых языков GLR-алгоритм выглядит наиболее предпочтительно.

Удобным способом формального определения грамматики, элементов и атрибутов языка программирования является расширенная нормальная форма Бэкуса-Наура(EBNF)~\cite{ISOEBNF}. На практике, использование EBNF позволяет упростить и сократить описание языка, сделать его более понятным. Кроме того, документация по языку, как правило, содержит конструкции EBNF. Однако многие современные инструменты не поддерживают EBNF-конструкции.

При работе с инструментом пользователь ожидает получить результат описанный в терминах заданной им грамматики. Это выдвигает дополнительные требования к алгоритму. В случае, если входная грамматика была каким-либо образом преобразована, например с целью раскрыть конструкции EBNF, то появляется необходимость в построении "обратного" \ преобразования. Это преобразование должно "перевести" \ результат обратно в термины входной грамматики. Такие преобразования  требуют дополнительных ресурсов и усложняют инструмент. Поэтому наиболее предпочтительными является алгоритмы, работающие без дополнительных преобразований грамматики.

В рамках данной работы ставится цель разработки прототипа генератора анализаторов, позволяющего работать с неоднозначными расширенными контекстно-свободными грамматиками, предоставляющего, в то же время, ставшее привычными средство автоматизации трансляции: дополнение грамматики атрибутами.
